\usepackage{booktabs}
\usepackage{pdfpages}
\usepackage{env/psl-cover}
\usepackage{rotating}
\usepackage{fancyhdr}
\usepackage{bibentry}

% abstract for each chapter
\newenvironment{chapabstract}[1]%
{\leftskip1in\textbf{#1 }\itshape}%

%%% Fancy Header %%%%%%%%%%%%%%%%%%%%%%%%%%%%%%%%%%%%%%%%%%%%%%%%%%%%%%%%%%%%%%%%%%
% Fancy Header Style Options

\pagestyle{fancy}                       % Sets fancy header and footer
\fancyfoot{}                            % Delete current footer settings

%\renewcommand{\chaptermark}[1]{         % Lower Case Chapter marker style
%  \markboth{\chaptername\ \thechapter.\ #1}}{}} %

%\renewcommand{\sectionmark}[1]{         % Lower case Section marker style
%  \markright{\thesection.\ #1}}         %

\fancyhead[LE,RO]{\bfseries\thepage}    % Page number (boldface) in left on even
% pages and right on odd pages
\fancyhead[RE]{\bfseries\nouppercase{\leftmark}}      % Chapter in the right on even pages
\fancyhead[LO]{\bfseries\nouppercase{\rightmark}}     % Section in the left on odd pages

\author{\mbox{Héctor CLIMENTE GONZÁLEZ}}

\institute{MINES ParisTech}
\doctoralschool{Ingénierie des Systèmes, Matériaux, Mécanique, Énergétique}{621}
\specialty{Bio-informatique}
\date{4 Février 2020}

%% cotutelle
% \entitle{}
% \otherinstitute{Institut Curie}
% \logootherinstitute{logo-institute}

\jurymember{1}{Nadine ANDRIEU}{Mme., Institut Curie}{Présidente}
\jurymember{2}{Kristel VAN STEEN}{Mme., Université de Liège}{Rapporteuse}
\jurymember{3}{Antonio RAUSELL}{M., Imagine Institute}{Rapporteur}
\jurymember{4}{Laura FURLONG}{Mme., Pompeu Fabra University}{Examinatrice}
\jurymember{5}{Véronique STOVEN}{Mme., MINES ParisTech}{Directrice de thèse}
\jurymember{6}{Chloé-Agathe AZENCOTT}{Mme., MINES ParisTech}{Co-encadrante}

\frabstract{
Cette thèse s'intéresse à un ensemble de méthodes utilisées pour identifier les causes génétiques de maladies complexes. Les méthodes d'association génome entier (GWAS), sont généralement utilisées pour étudier des associations univariées, tandis que les méthodes d'association d'interactions génome entier (GWAIS) prennent en considération des interactions entre facteurs génétiques (ou épistasie). Cependant, ces deux approches présentent plusieurs défis, parmi lesquels leur faible puissance statistique, la difficulté de leur interprétation, ainsi que les choix arbitraires qui doivent être faits à différentes étapes de ces études. Dans cette thèse, j'étudie comment l'utilisation de réseaux biologiques permet de répondre à ces défis et faciliter la découverte de nouveaux biomarqueurs. Les réseaux biologiques permettent en effet d'incorporer des connaissances a priori aux analyses statistiques, et de considérer chaque polymorphisme d'un seul nucléotide (SNP) et chaque gène dans leur contexte biologique. En analysant deux jeux de données, un sur le cancer du sein et l'autre sur les maladies chroniques inflammatoires de l'intestin, je montre comment l'utilisation de réseaux biologiques permet de mettre à jour de nouveaux mécanismes de susceptibilité. Ceux-ci impliquent des SNPs individuels, ainsi que des groupes de SNPs en épistasie d'ordre deux ou plus. Je montre aussi comment l'incorporation de réseaux biologique dans les GWAS et GWAIS permet d'améliorer l'interprétabilité des résultats et de produire des hypothèses biologiques convaincantes.
}

\enabstract{
This thesis tackles methodologies to identify the genetic causes of complex diseases. This is usually done via genome-wide association studies (GWAS), when univariate association is studied, and genome-wide association interaction studies, when interactions between genetic factors (or epistasis) are considered (GWAIS). However, both settings present some challenges, namely low statistical power, difficult interpretation, and arbitrary choices at multiple points of the study. In this thesis I study how a framework that uses biological networks can help overcome these issues and boost biomarker discovery. This is done by incorporating prior knowledge into the statistical analysis and putting every single nucleotide polymorphism (SNP) and gene in relation to their biological context. By analyzing two datasets, on breast cancer and inflammatory bowel disease, I demonstrate the utility of networks to discover new mechanisms of susceptibility. These involve individual SNPs, as well as groups of SNPs in epistasis, two-way and higher. I also show how including networks in GWAS and GWAIS boosts the interpretability of the results and produces compelling biological hypotheses.
}

\frkeywords{ GWAS; Études d'association génome entier; Réseaux; Épistasie; Apprentissage statistique; Cancer du sein; Maladies chroniques inflammatoires de l'intestin }
\enkeywords{ GWAS; Network; Epistasis; Machine learning; Inflammatory bowel disease; Breast cancer }

\frontmatter